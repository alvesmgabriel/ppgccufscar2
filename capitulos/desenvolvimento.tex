% ------------------------------------------------------------------------
% ------------------------------------------------------------------------
% desenvolvimento.tex, criado em 11-04-2016
% autor: gabriel marcelino alves
% descricao: arquivo de exemplo de desenvolvimento do trabalho
%
% ------------------------------------------------------------------------ 
% ------------------------------------------------------------------------

\chapter{Desenvolvimento}
Segundo \citeonline{2007_montebelo}, \lipsum[2].

\section{Primeira seção}
\lipsum[4]

\subsection{Subseção 1}
Lorem ipsum dolor sit amet, consectetuer adipiscing elit. Ut purus elit, vestibulum ut, placerat ac,
adipiscing vitae, felis. Curabitur dictum gravida mauris. Nam arcu libero, nonummy eget, consectetuer
id, vulputate a, magna. Donec vehicula augue eu neque. Pellentesque habitant morbi tristique senectus
et netus et malesuada fames ac turpis egestas. Mauris ut leo. Cras viverra metus rhoncus sem. Nulla
et lectus vestibulum urna fringilla ultrices \cite{2007_montebelo}.

\subsection{Subseção 2}
\lipsum[3]

\subsubsection{Subsubseção 1}
Pode-se observar, ainda, que
\begin{citacao}
	as citações diretas, no texto, devem ser usadas utilizando o ambiente \texttt{citacao}. Dessa forma é possível reproduzir a citação com mais de três linhas com recuo de 4cm da margem esquerda e com letra menor que a do texto utilizado e sem aspas como determina a norma ABNT NBR10520:2002.
\end{citacao}

\lipsum[2]

A Tabela \ref{tabela-ibge} é um exemplo de tabela no padrão IBGE cujo comando \texttt{IBGEtab} é fornecido pela classe abnTeX2.
\begin{table}[htb]
	\IBGEtab{
		\caption{Tabela no padrão IBGE.}%
		\label{tabela-ibge}
	}{
	\begin{tabular}{ccc}
		\toprule
		Nome & Nascimento & Documento \\
		\midrule \midrule
		Maria da Silva & 11/11/1111 & 111.111.111-11 \\
		\bottomrule
	\end{tabular}
	}{
		\fonte{Produzido pelos autores}
	}
\end{table}
