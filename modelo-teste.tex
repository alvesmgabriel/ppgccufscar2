\documentclass[qualit]{ppgccufscar2}

% ------------------------------------------------------------------------------
% Pacotes adicionais
\usepackage{lipsum}		% Dummy text

% ------------------------------------------------------------------------------
% Configurações gerais do documento
\titulo{Modelo para elaboração de dissertações e teses do PPGCC-UFSCar com base na suíte abn\TeX2}
\autor{Nome do aluno da Pós-Graduação}
\orientador[Orientadora:]{Profa. Dra. Fulana de Tal}
\coorientador[Co-orientador:]{Prof. Dr. Beltrano da Silva}
\areaconcentracao{Processamento de Imagens e Sinais}
\data{Abril, 2017}

% ------------------------------------------------------------------------------
% Inicio do documento
\begin{document}
	
% -- imprime capa e folha de rosto (elementos obrigatórios)
\imprimircapa
\imprimirfolhaderosto

% ------------------------------------------------------------------------------
% Elementos pre-textuais. É obrigatório colocar o comando \pretextual
\pretextual

% ------------------------------------------------------------------------
% ------------------------------------------------------------------------
% ficha.tex, criado em 11-04-2016
% autor: gabriel marcelino alves
% descricao: arquivo de exemplo de ficha catalográfica do trabalho
%
% obs.: Exemplo extraído do material da abnTeX2
% ------------------------------------------------------------------------ 
% ------------------------------------------------------------------------
%TODO: Verificar uma maneira de automatizar a impressão da ficha (algo como \imprimirficha)
%TODO: Definir variáveis para as palavras-chaves
%TODO: Verificar o que significa ': il. (algumas color.) ; 30 cm.'

\begin{fichacatalografica}
	\sffamily
	\vspace*{\fill}					% Posição vertical
	\begin{center}					% Minipage Centralizado
		\fbox{\begin{minipage}[c][8cm]{13.5cm}		% Largura
				\small
				\imprimirautor
				%Sobrenome, Nome do autor
				
				\hspace{0.5cm} \imprimirtitulo  / \imprimirautor. --
				\imprimirlocal, \imprimirdata-
				
				\hspace{0.5cm} \pageref{LastPage} p. : il. (algumas color.) ; 30 cm.\\
				
				\hspace{0.5cm} \imprimirorientadorRotulo~\imprimirorientador\\
				
				\hspace{0.5cm}
				\parbox[t]{\textwidth}{\imprimirtipotrabalho~--~\imprimirinstituicao,
					\imprimirdata.}\\
				
				\hspace{0.5cm}
				1. Palavra-chave1.
				2. Palavra-chave2.
				2. Palavra-chave3.
				I. Orientador.
				II. Universidade xxx.
				III. Faculdade de xxx.
				IV. Título 			
			\end{minipage}}
		\end{center}
	\end{fichacatalografica}
					% Ficha catalográfica - obrigatório
\input{pretextuais/errata}					% Errata - opcional
\input{pretextuais/aprovacao}				% Folha de aprovação - obrigatório
% ------------------------------------------------------------------------
% ------------------------------------------------------------------------
% dedicatoria.tex, criado em 11-04-2016
% autor: gabriel marcelino alves
% descricao: arquivo de exemplo de dedicatória do trabalho
%
% ------------------------------------------------------------------------ 
% ------------------------------------------------------------------------

\begin{dedicatoria}
	\vspace*{\fill}
	\centering
	\noindent
	\textit{Este trabalho é dedicado aos meus pais.} \vspace*{\fill}
\end{dedicatoria}				% Dedicatória - opcional
% ------------------------------------------------------------------------
% ------------------------------------------------------------------------
% agradecimentos.tex, criado em 11-04-2016
% autor: gabriel marcelino alves
% descricao: arquivo de exemplo de agradecimentos do trabalho
%
% ------------------------------------------------------------------------ 
% ------------------------------------------------------------------------

\begin{agradecimentos}
	Agradeço a todos que direta, ou indiretamente, contribuíram para o sucesso deste trabalho.	
\end{agradecimentos}			% Agradecimentos - opcional
% ------------------------------------------------------------------------
% ------------------------------------------------------------------------
% epigrafe.tex, criado em 11-04-2016
% autor: gabriel marcelino alves
% descricao: arquivo de exemplo de epigrafe do trabalho
%
% ------------------------------------------------------------------------ 
% ------------------------------------------------------------------------

\begin{epigrafe}
	\vspace*{\fill}
	\begin{flushright}
		\textit{Sempre em frente, confiante, na patinha do elefante!\\
			(Zu, "O Zoo da Zu")}\\
		
		\vspace{3cm}
				
		\textit{Essa é a melhor experiência de todas!\\
			(Luna, "O Show da Luna")}
	\end{flushright}	
\end{epigrafe}
				% Epígrafe - opcional
\input{pretextuais/resumo}					% Resumo em português - obrigatório
% ------------------------------------------------------------------------
% abstract.tex, criado em 11-04-2016
% autor: gabriel marcelino alves
% descricao: arquivo de exemplo de resumo em inglês do trabalho
%
% obs.: Exemplo extraído e adaptado do material da abnTeX2
% ------------------------------------------------------------------------ 
% ------------------------------------------------------------------------

\setlength{\absparsep}{18pt} % ajusta o espaçamento dos parágrafos do resumo
\begin{resumo}[Abstract]
	\begin{otherlanguage*}{english}
		This is the english abstract.
		
		\vspace{\onelineskip}
		
		\noindent 
		\textbf{Keywords}: latex. abntex. text editoration.
	\end{otherlanguage*}
\end{resumo}
				% Resumo em inglês - obrigatório
% Lista de figuras
% Lista de tabelas
% Lista de quadros
% Lista de algoritmos
% ------------------------------------------------------------------------
% ------------------------------------------------------------------------
% siglas.tex, criado em 12-04-2016
% autor: gabriel marcelino alves
% descricao: arquivo de exemplo de lista de siglas
%
% ------------------------------------------------------------------------ 
% ------------------------------------------------------------------------

\begin{siglas}
	\item[ABNT] Associação Brasileira de Normas Técnicas
	\item[abnTeX] ABsurdas Normas para TeX
\end{siglas}
					% Lista de siglas
% ------------------------------------------------------------------------
% ------------------------------------------------------------------------
% simbolos.tex, criado em 12-04-2016
% autor: gabriel marcelino alves
% descricao: arquivo de exemplo de lista de simbolos
%
% ------------------------------------------------------------------------ 
% ------------------------------------------------------------------------

\begin{simbolos}
	\item[$ \Gamma $] Letra grega Gama
	\item[$ \Lambda $] Lambda
	\item[$ \zeta $] Letra grega minúscula zeta
	\item[$ \in $] Pertence
\end{simbolos}
				% Lista de símbolos

\inserirlistadefiguras						% Insere lista de figuras - opcional
\inserirlistadetabelas						% Insere lista de tabelas - opcional
\inserirsumario								% Insere sumário no documento - obrigatório
% ------------------------------------------------------------------------------
% Elementos textuais. É obrigatório colocar o comando \textual antes de iniciar.
\textual

% ------------------------------------------------------------------------
% ------------------------------------------------------------------------
% introducao.tex, criado em 11-04-2016
% autor: gabriel marcelino alves
% descricao: arquivo de exemplo de introdução do trabalho
%
% ------------------------------------------------------------------------ 
% ------------------------------------------------------------------------

\chapter{Introdução}

\listfigurename

\lipsum[3]

\section{Contextualização}
A Figura \ref{fig:fig01} trata do \lipsum[1-3]

\begin{figure}[!htb]
	\label{fig:fig01}
	\begin{center}
		\includegraphics[scale=0.60]{capitulos/imgs/abntex2-modelo-img-marca.pdf}
	\end{center}
	\caption{Exemplo de inclusão de imagem a partir de arquivo PDF. Fonte: \url{http://www.anbtex.net.br}}
\end{figure}

\lipsum[3]
		% Capítulo de introdução
% ------------------------------------------------------------------------
% ------------------------------------------------------------------------
% desenvolvimento.tex, criado em 11-04-2016
% autor: gabriel marcelino alves
% descricao: arquivo de exemplo de desenvolvimento do trabalho
%
% ------------------------------------------------------------------------ 
% ------------------------------------------------------------------------

\chapter{Desenvolvimento}
Segundo \citeonline{2007_montebelo}, \lipsum[2].

\section{Primeira seção}
\lipsum[4]

\subsection{Subseção 1}
Lorem ipsum dolor sit amet, consectetuer adipiscing elit. Ut purus elit, vestibulum ut, placerat ac,
adipiscing vitae, felis. Curabitur dictum gravida mauris. Nam arcu libero, nonummy eget, consectetuer
id, vulputate a, magna. Donec vehicula augue eu neque. Pellentesque habitant morbi tristique senectus
et netus et malesuada fames ac turpis egestas. Mauris ut leo. Cras viverra metus rhoncus sem. Nulla
et lectus vestibulum urna fringilla ultrices \cite{2007_montebelo}.

\subsection{Subseção 2}
\lipsum[3]

\subsubsection{Subsubseção 1}
Pode-se observar, ainda, que
\begin{citacao}
	as citações diretas, no texto, devem ser usadas utilizando o ambiente \texttt{citacao}. Dessa forma é possível reproduzir a citação com mais de três linhas com recuo de 4cm da margem esquerda e com letra menor que a do texto utilizado e sem aspas como determina a norma ABNT NBR10520:2002.
\end{citacao}

\lipsum[2]

	% Capítulo de desenvolvimento
% ------------------------------------------------------------------------
% ------------------------------------------------------------------------
% conclusão.tex, criado em 11-04-2016
% autor: gabriel marcelino alves
% descricao: arquivo de exemplo de conclusão do trabalho
%
% ------------------------------------------------------------------------ 
% ------------------------------------------------------------------------
\chapter{Conclusão}
\lipsum
			% Capítulo de conclusão

% ------------------------------------------------------------------------------
% Elementos pós-textuais. É obrigatório colocar o comando \postextual antes de iniciar.
\postextual

% ----------------------------------------------------------
% Referências bibliográficas
\bibliography{refs}



\end{document}